%%% Template originaly created by Karol Kozioł (mail@karol-koziol.net) and modified for ShareLaTeX use

\documentclass[a4paper,11pt]{article}

\usepackage[T1]{fontenc}
\usepackage[utf8]{inputenc}
\usepackage{graphicx}
\usepackage{xcolor}

\renewcommand\familydefault{\sfdefault}
\usepackage{tgheros}
\usepackage[defaultmono]{droidmono}

\usepackage{amsmath,amssymb,amsthm,textcomp,pdfpages,pdflscape,hyperref}
\usepackage{enumerate}
\usepackage{multicol}
\usepackage{tikz}

\usepackage{geometry}
\geometry{left=25mm,right=25mm,%
bindingoffset=0mm, top=20mm,bottom=20mm}


\linespread{1.3}

\newcommand{\linia}{\rule{\linewidth}{0.5pt}}

% custom theorems if needed
\newtheoremstyle{mytheor}
    {1ex}{1ex}{\normalfont}{0pt}{\scshape}{.}{1ex}
    {{\thmname{#1 }}{\thmnumber{#2}}{\thmnote{ (#3)}}}

\theoremstyle{mytheor}
\newtheorem{defi}{Definition}

% my own titles
\makeatletter
\renewcommand{\maketitle}{
\begin{center}
\vspace{2ex}
{\huge \textsc{\@title}}
\vspace{1ex}
\\
\linia\\
\@author \hfill \@date
\vspace{4ex}
\end{center}
}
\makeatother
%%%

% custom footers and headers
\usepackage{fancyhdr}
\pagestyle{fancy}
\lhead{}
\chead{}
\rhead{}
\lfoot{Trabajo práctico \textnumero{} 1}
\cfoot{}
\rfoot{Página \thepage}
\renewcommand{\headrulewidth}{0pt}
\renewcommand{\footrulewidth}{0pt}
%

% code listing settings
\usepackage{listings}
\lstset{
    language=Python,
    basicstyle=\ttfamily\small,
    aboveskip={1.0\baselineskip},
    belowskip={1.0\baselineskip},
    columns=fixed,
    extendedchars=true,
    breaklines=true,
    tabsize=4,
    prebreak=\raisebox{0ex}[0ex][0ex]{\ensuremath{\hookleftarrow}},
    frame=lines,
    showtabs=false,
    showspaces=false,
    showstringspaces=false,
    keywordstyle=\color[rgb]{0.627,0.126,0.941},
    commentstyle=\color[rgb]{0.133,0.545,0.133},
    stringstyle=\color[rgb]{01,0,0},
    numbers=left,
    numberstyle=\small,
    stepnumber=1,
    numbersep=10pt,
    captionpos=t,
    escapeinside={\%*}{*)}
}

%%%----------%%%----------%%%----------%%%----------%%%

\begin{document}

\title{Trabajo práctico \textnumero{} 1}

\author{Juan Andres Knebel, Universidad de Buenos Aires}

\date{28/05/2020}

\maketitle

\section*{Diseñando una visualización}
La visualización esta presentada y organizada en dos partes. La parte izquierda responde a la primer pregunta y la derecha la restante.

Se utilizaron colores pasteles con fondos claros (blanco y gris) para las barras y líneas, y una escala de rojos con diferentes tamaños sobre los departamentos. La elección tan diferente de estos dos usos de colores es para resaltar sobre el mapa geográfico aquellos departamentos afectados.

En esta visualización se dejaron de lado los datos acerca de la finalización de los incendios.
\section{¿Como se distribuyen los incendios en el país?}
A partir de esta pregunta quise mostrar de una manera eficaz y rápida cuales fueron las provincias y departamentos afectados por los incendios. Reforzando al mismo tiempo el daño ocasionado por los mismos (en este caso medido en hectáreas afectadas) que se puede observar en el mapa de la Argentina con sus puntos en una escala de rojos, dónde los puntos más gruesos y más oscuros representan mayor superficie afectada.
A su lado se muestra una evolución a través de los años de las superficies afectadas agrupado por provincia.

\section{¿Hay relación entre la cantidad de focos de incendios y las superficies afectadas?}
Los dos gráficos de la derecha muestran simultáneamente y con dos escalas diferentes las dos variables cantidad de focos y superficie afectada. La de arriba muestra el paso del tiempo y como crecen y decrecen, mientras que la restante exhibe los mismos datos históricos y agrupados por provincia.
\section*{Herramientas y datos extra}
Para los gráficos se utilizó python 3.7 con las librerias pandas, geopandas, matplotlib. Se usaron datos extras geográficos de la Argentina obtenidos desde \url{https://www.ign.gob.ar}.

\begin{landscape}
\includegraphics[scale=0.75]{graficos.pdf}
%\includegraphics[scale=0.75,angle=270]{graficos.pdf}
\end{landscape}
\end{document}
