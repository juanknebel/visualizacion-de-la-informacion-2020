%%%%%%%%%%%%%%%%%%%%%%%%%%%%%%%%%%%%%%%%%
% fphw Assignment
% LaTeX Template
% Version 1.0 (27/04/2019)
%
% This template originates from:
% https://www.LaTeXTemplates.com
%
% Authors:
% Class by Felipe Portales-Oliva (f.portales.oliva@gmail.com) with template 
% content and modifications by Vel (vel@LaTeXTemplates.com)
%
% Template (this file) License:
% CC BY-NC-SA 3.0 (http://creativecommons.org/licenses/by-nc-sa/3.0/)
%
%%%%%%%%%%%%%%%%%%%%%%%%%%%%%%%%%%%%%%%%%

%----------------------------------------------------------------------------------------
%	PACKAGES AND OTHER DOCUMENT CONFIGURATIONS
%----------------------------------------------------------------------------------------

\documentclass[
	12pt, % Default font size, values between 10pt-12pt are allowed
	%letterpaper, % Uncomment for US letter paper size
	spanish, % Uncomment for Spanish
]{fphw}

\usepackage{setspace}
%\singlespacing
\onehalfspacing
%\doublespacing

% Template-specific packages
\usepackage[utf8]{inputenc} % Required for inputting international characters
\usepackage[T1]{fontenc} % Output font encoding for international characters
\usepackage{mathpazo} % Use the Palatino font
\usepackage{hyperref}
\hypersetup{
    unicode=false,          % non-Latin characters in Acrobat’s bookmarks
    pdftoolbar=true,        % show Acrobat’s toolbar?
    pdfmenubar=true,        % show Acrobat’s menu?
    pdffitwindow=false,     % window fit to page when opened
    pdfstartview={FitH},    % fits the width of the page to the window
    pdftitle={My title},    % title
    pdfauthor={Author},     % author
    pdfsubject={Subject},   % subject of the document
    pdfcreator={Creator},   % creator of the document
    pdfproducer={Producer}, % producer of the document
    pdfkeywords={keyword1, key2, key3}, % list of keywords
    pdfnewwindow=true,      % links in new PDF window
    colorlinks=true,       % false: boxed links; true: colored links
    linkcolor=red,          % color of internal links (change box color with linkbordercolor)
    citecolor=green,        % color of links to bibliography
    filecolor=magenta,      % color of file links
    urlcolor=cyan           % color of external links
}
\usepackage{caption}
%\captionsetup{labelformat=empty}


\usepackage{graphicx} % Required for including images

\usepackage{booktabs} % Required for better horizontal rules in tables

\usepackage{listings} % Required for insertion of code

\usepackage{enumerate} % To modify the enumerate environment

\usepackage{amsmath}

%----------------------------------------------------------------------------------------
%	ASSIGNMENT INFORMATION
%----------------------------------------------------------------------------------------

\title{Integridad Visual} % Assignment title

\author{Juan Andres Knebel} % Student name

\date{\today} % Due date

\institute{Universidad de Buenos Aires \\ Maestría en Explotación de Datos y Descubrimiento del Conocimiento} % Institute or school name

\class{Visualización de la información - Trabajo práctico \#2} % Course or class name

%\professor{Dr. Albert Einstein} % Professor or teacher in charge of the assignment

%----------------------------------------------------------------------------------------

\begin{document}

\maketitle % Output the assignment title, created automatically using the information in the custom commands above

%----------------------------------------------------------------------------------------
%	ASSIGNMENT CONTENT
%----------------------------------------------------------------------------------------

\section*{Ejemplos de gráficas}

%------------------------------------------------

\subsection*{Primera elección}

\textit{Un gráfico está distorsionado si la representación visual que presenta no se condice con la representación numérica.}

\begin{figure}[h]
\begin{center}
	\includegraphics[width=0.7\columnwidth,keepaspectratio]{lanacion-plasticos.png}
	\caption*{Fuente: \href{https://www.lanacion.com.ar/sociedad/6-graficos-para-entender-el-problema-del-plastico-nid1968639}{Diario La Nación}}
\end{center}
\end{figure}

$$ \textbf{Factor de Mentira}\ =\ \dfrac{proporci\acute on\ en\ el\ gr\acute afico}{proporci\acute on\ en\ los\ datos} $$

Se muestra un incremento que va desde los 8.75 millones hasta los 275 millones, o lo que es lo mismo una variación cercana al 3042\%.

$$ \dfrac{275 - 8.75}{8.75} \times 100  \approx 3042$$

El lado del cuadrado que representa los 275 millones tiene un longitud de 10.5 cm, siendo su área de 110.25 $cm^{2}$.

El lado del cuadrado correspondiente a los 8.75 millones es de 2.9 cm, resultando su área en 8.41 $cm^{2}$. Dando como resultado una variación aproximada del 1210\%

$$ \dfrac{110.25 - 8.41}{8.41} \times 100  \approx 1210$$


$$ \textbf{Factor de Mentira}\ = \dfrac{1210}{3042} \approx \boxed{0.39} $$

Siendo que el factor de mentira es de $\textit{0.39} < 0.95$, se puede decir que el gráfico no está representando los datos adecuadamente. También se puede ver que como $log(0.39) < 0$ entonces se está incurriendo en un error por reducción.

\subsection*{Segunda elección}
\textit{En las series de tiempo monetarias, casi siempre es mejor usar unidades estandarizadas en lugar de nominales.}

\begin{figure}[h]
\begin{center}
	\includegraphics[width=0.7\columnwidth,keepaspectratio]{lanacion-salirio-minimo.png}
	\caption*{Fuente: \href{https://www.lanacion.com.ar}{Diario La Nación}}
\end{center}
\end{figure}

Se muestra la variación del salario mínimo, vital y móvil a través de los últimos 7 años en la Argentina utilizando el valor nominal del peso argentino. Teneindo en cuenta que en los últimos 7 años el país pasó y pasa por períodos de inflación por encima del 20\% y fuertes devaluaciones con respecto al dólar, el gráfico presentado no aporta nada por si mismo.

\newpage
%----------------------------------------------------------------------------------------

\section*{La peor visualización del mundo}



\end{document}
